\documentclass{IEEEtran}
\usepackage{amssymb}
%\usepackage[francais]{babel}
%\usepackage[utf8]{inputenc}
\usepackage{amsmath}
\usepackage{amscd}
\usepackage{hyperref}
\usepackage{verbatim}
\usepackage{subfigure}
\usepackage{tikz}
\usepackage{proof}
\usepackage{alltt}
\usepackage{color}
\usepackage{xspace}
\usepackage{tikz,ifthen}
\usepackage{fancybox}
\usepackage{wrapfig}
\usepackage{multirow}
\usepackage{listings}
\usepackage{rotating}
\usepackage{xcolor,colortbl}
\usepackage{epstopdf}
\usepackage[textsize=small,colorinlistoftodos]{todonotes}
\pagestyle{plain}
%
\usepackage{makeidx}  % allows for indexgeneration
\usepackage{algpseudocode}
\usepackage{algorithm}


\newcommand{\ft}[1]{{\small\textsf{#1}}}
\newcommand{\uppaal}{\textsc{Uppaal}\xspace}
\newcommand{\uppaalsmc}{\textsc{Uppaal SMC}\xspace}
\newcommand{\R}{\mathbb{R}_{\ge 0}}

\newcommand{\CR}{\emph{WCRA}_c^r}
\newcommand{\CW}{\emph{WCRA}_c^w}
\newcommand{\MR}{\emph{WCRA}_m^r}
\newcommand{\MW}{\emph{WCRA}_m^w}

\newtheorem{definition}{Definition}
%\newtheorem{theorem}{Theorem}
%\newtheorem{property}{Property}



\IEEEoverridecommandlockouts


\title{Performance-aware Scheduling of Multicore Time-critical Systems \thanks{This work has been supported by the NFFP6 Project.}}


\author{
\IEEEauthorblockN{Jalil Boudjadar, Jin Hyun Kim, Simin Nadjm-Tehrani} \\
\IEEEauthorblockA{Department of Computer and Information Science \\ Link\"oping University, Sweden} 
}


\begin{document}
\maketitle


\begin{abstract}
Multicore platforms are getting widely used in the deployment of embedded systems due to their computing capacity. Memory interference represents a challenge in leveraging the performance of multicore platforms and may lead to drastic degradations of the platform performance, in particular for memory-intensive systems. This paper explores performance-driven schedulability of  multicore systems by evaluating the performance when changing some design parameters (scheduling policies). The model-based framework we build enables to analyze the performance of multicore time-critical systems running core-centric and memory-centric scheduling policies. The system architecture we consider consists of a set of cores having each a local cache and sharing the cache level L2 and DRAM. The metrics we use to compare the performance achieved by different configurations of a system are: 1) cores utilization; and 2) the maximum delay per access request to shared cache and DRAM. Our framework, realized using Uppaal, can be viewed as an engineering tool, to be used during design stages, to identify the scheduling policies achieving  better performance for a given system while maintaining the system schedulability. As a proof of concept, we analyze 2 different cases studies and compare the outcomes.
\end{abstract}



\section{Introduction}
\input{introduction}

\section{Related Work}
\label{sec:relatedwork}
Performance represents a key factor in comparing and identifying multicore platforms which are more adequate for given users and applications. The selection of performance metrics depends on the resources and services a system runs as performance usually resources utilization and system outputs. The most common performance analysis techniques for multicore systems are analytical modeling \cite{Heechul2015,Kim14}; and simulation and measurement \cite{***,DBLP:journals/corr/TeodoroKAKFS15,10.1109/MS.2005.102}. 

\cite{10.1109/MS.2005.102} presents a survey on performance analysis techniques for COTS-based systems and discusses the trade-off between the design models used to capture systems to be analyzed and the performance attributes to be measured. Delays and throughput have been identified as main metrics among other performance attributes such as resource contention. In a similar way, \cite{Rakhee2014} presents a set of performance metrics (throughput, energy consumption, memory contention, \textit{etc}) and criteria (time, cost, accuracy, \textit{etc}) to compare a set of multicore platforms and architectures. A conclusion driven by the author is that performance metrics need to be chosen based on the purpose and efficiency of the analysis while taking advantage of the multicore platform architectures. 

The authors of \cite{Heechul2015,Kim14} introduce analytical frameworks to calculate memory interference delays in multicore systems. The interference of the request under analysis is calculated based on inter and intra-bank interference as well as \textit{row-opening} (loading data from a row to a row-buffer) and \textit{precharge} (moving data back from a row-buffer to a row). Each of these elements is calculated separately. Since each request is analyzed separately, a relevant question is how these analytical frameworks deal with memory-intensive systems where each process performs thousands of memory requests.


The authors of \cite{DBLP:journals/corr/TeodoroKAKFS15} perform an experimental comparison of different multicore architectures in terms of performance, and propose new scheduling strategies to improve the performance of multicore systems. The platforms considered consist each of a set of heterogeneous types of cores having different frequencies. The execution model is implemented in terms of code annotation using OpenMP. Two scheduling strategies have been discussed: Performance-Aware Dequeue Adapted Scheduler (PADAS) and Performance-Aware Multiqueue Scheduler (PAMS). In PADAS, tasks are sorted out in one ready queue (global) based on the acceleration (speedup) expected by each task, where tasks having higher acceleration have priority over others. In this context, acceleration comes from the use of co-processors. A task is assigned to a core based on whether the core frequency satisfies the task acceleration or not. In PAMS, local scheduling is adopted for each core type so that tasks having certain range of acceleration are sorted in the appropriate queue according to a descending order of the acceleration. This framework relies on an efficient task-to-core mapping to achieve better performance. Tasks response time is the performance metric used to compare systems when using different scheduling policies: PADAS, PAMS, Heterogeneous Earliest Finish Time (HEFT) and First Come First Served (FCFS).  
%
%However, having static frequency for cores can lead to cores available while tasks having non compatible acceleration are waiting to be scheduled, which is could be a non efficient use of the platform. 
Our work alines with \cite{DBLP:journals/corr/TeodoroKAKFS15}, to certain extent, however it differs by using system benchmarks rather than codes. Moreover, by adopting the AMP scheduling strategy we assume that identifying the optimal task-to-core mapping is beyond the scope of this paper.

\cite{5762713} evaluation and optimization of the performance

\cite{Subramanian13} Providing performance predictability and improving memory interference 

\cite{SHARMA2014544} The tool also provides the possibility of implementing different scheduling and mapping algorithms for multi-core process systems and evaluating the performance when changing various design parameters.

%Analyzing the schedulability of an application while considering an abstraction of the execution platform leads necessarily to underestimate the system requirements (workload) in terms of resources. This may cause serious deficits when deploying the system on a concrete platform. 

The analysis of multicore real-time systems has attracted a lot of attention recently. 

%Given a multicore execution platform, including processing units, local caches and a shared memory, we formally analyze the system schedulability. Moreover, our framework can easily be used to estimate the concrete execution time, including cache access and blobking/access time to the shared memory, of the application tasks.    



\section{Background}
\input{background}

\section{Performance Metrics}
\input{performance}

\section{System Model Description}
\input{modeling}


\section{Performance Analysis}
\input{analysis}

\section{Case Study}
\input{casestudy}


\section{conclusion}
\input{conclusion}


\bibliographystyle{abbrv}
\bibliography{references}

%\listoftodos

\end{document}
